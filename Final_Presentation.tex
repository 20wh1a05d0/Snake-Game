\documentclass[14pt]{beamer}
\title{BVRIT Hyderabad College of Engineering for Women}
\subtitle {Team - 05}
\date{\today}
\author[Bvrith]{SNAKE GAME}
\usefonttheme{serif}
\usepackage{bookman}
\usepackage{hyperref}
\usepackage[T1]{fontenc}
\usepackage{graphicx}
\usecolortheme{orchid}
\beamertemplateballitem
\begin{document}
    \begin{frame}
        \titlepage
    \end{frame}
\begin{frame}
	\frametitle{Team Members:}
        \begin{itemize}
	 \item Ch.Sriniketha: 20WH1A1255: IT 
	\item  T.Yasaswini: 20WH1A0518: CSE
	 \item  D.Kavya Sri: 20WH1A0401: ECE 
	\item B.Swetha: 20WH1A04B3: ECE
	\item Y.Monisha: 20WH1A05D0: CSE
\end{itemize}
    \end{frame}

    \begin{frame}
	\frametitle{Introduction}
        \begin{itemize}
	    \item Snake game is one of the most popular arcade games of all the time.The main objective of the player playing the game is to catch the maximum number of fruits without hitting the wall or itself.
	\end{itemize}
    \end{frame}

    \begin{frame}
	\frametitle{Approach}
	\begin{itemize}
	   \item Understanding the problem statement.
	   \item Going through the pygame documentation. 
	    \item Following the given steps in writing the code:
	    \begin{itemize}	 
             \item  Installing Pygame
             \item Create the screen
             \item Create the snake
             \item Moving the snake
             \item Game over when the snake hits the walls
             \item Adding the fruits
             \item Increasing the length of the snake
             \item Displaying the score
	\end{itemize}
	\end{itemize}    
	\end{frame}

	\begin{frame}
	\frametitle{Technical stats}
	\begin{itemize}
	\item Using command prompt to write the code
	\item Using Latex to make the presentations
	\item Gitlab to keep track of the work
	\end{itemize}
	\end{frame}
 

    \begin{frame}
        \frametitle{Learnings}
	\begin{itemize}
	    \item  The usage of Latex software.
	    \item Got to know what is Gitlab and how to use it.
              \item About the various functions and events in the pygame module.
		\item Controlling the movement of the objects on the screen.
	\end{itemize}
    \end{frame}
    \begin{frame}
	\frametitle{Challenges}
        \begin{itemize}
	    \item Faced difficulty in :
	   \begin{itemize}
		\item  Writing the code for the movement of the snake.
	          \item  Increasing the length of the snake .
		\item Adding sounds.
		\item Adding menu(Start,Continue,Quit options) to the game 
\end{itemize}    
\end{itemize}
    \end{frame}
\begin{frame}
	\frametitle{GIT Repo}
          \begin{figure}
	  \includegraphics[width = 8cm]{Git_Repo.jpg}
           \end{figure} 
    \end{frame}
\begin{frame}
	\frametitle{Screenshots of the project}
          \begin{figure}
	  \includegraphics[width = 8cm]{start_ss.jpg}
           \end{figure} 
    \end{frame}
\begin{frame}
          \begin{figure}
	  \includegraphics[width = 8cm]{ss1.jpg}
           \end{figure} 
    \end{frame}
\begin{frame}
          \begin{figure}
	  \includegraphics[width = 8cm]{ss2.jpg}
           \end{figure} 
    \end{frame}
\begin{frame}
          \begin{figure}
	  \includegraphics[width = 8cm]{end_ss.jpg}
           \end{figure} 
    \end{frame}
\begin{frame}
	\frametitle{Statistics}
        \begin{itemize}
	     \item Number of Lines of Code : 184
	     \item Number of Functions : 2
        \end{itemize}
    \end{frame}
    \begin{frame}
	\begin{center}
	     THANK YOU
	\end{center}
    \end{frame}
\end{document}

